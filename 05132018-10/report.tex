%!TEX program = xelatex
%!TEX TS-program = xelatex
%!TEX encoding = UTF-8 Unicode

\documentclass[a4paper]{article}
\usepackage[UTF8, heading = false, scheme = plain]{ctex}
\usepackage{graphicx}
\usepackage{cite}
\usepackage{geometry}
\geometry{left=2.0cm, right=2.0cm, top=2.5cm, bottom=2.5cm}
\usepackage[colorlinks,linkcolor=red,anchorcolor=blue,citecolor=green]{hyperref}
\usepackage{subfig}
\usepackage{caption}
\captionsetup{font={scriptsize}}

\renewcommand\figurename{图}

\makeatletter
\let\@afterindentfalse\@afterindenttrue
\@afterindenttrue
\makeatother
\setlength{\parindent}{2em}  

\linespread{1.4}
\setlength{\parskip}{0.5\baselineskip}

\title{学习汇报\\第十周}
\author{熊凯亚}
\date{\today}

\begin{document}
\maketitle

\section{本周工作}
由于音乐区块链项目时间比较紧迫,本周主要在做这个项目。我在项目里负责区块链节点底层parity的配置及其部署,因为需要使用Docker容器技术,所以本周花在这个方面的时间比较多。
\section{工作内容}
\subsubsection*{Centos 安装Docker-CE步骤}
\begin{enumerate}
\item 卸载已有版本(可能)
sudo yum remove docker docker-client docker-client-latest docker-common docker-latest docker-latest-logrotate docker-logrotate docker-selinux docker-engine-selinux docker-engine

\item 设置yum库 
sudo yum install -y yum-utils device-mapper-persistent-data lvm2
sudo yum-config-manager  --add-repo https://download.docker.com/linux/centos/docker-ce.repo

\item 安装 
sudo yum install docker-ce

\item 启动 
sudo systemctl start docker 

\item 安装Docker-compose
sudo curl -L https://github.com/docker/compose/releases/download/1.21.0/docker-compose-(uname -s)-(uname -m) -o /usr/local/bin/docker-compose

\item 赋予执行权限 
sudo chmod +x /usr/local/bin/docker-compose
\end{enumerate}

\subsubsection*{Ubuntu安装Docker-CE步骤} 
\begin{enumerate}
\item 卸载旧版本 
sudo apt-get remove docker docker-engine docker.io

\item 更新apt索引 
sudo apt-get update

\item 使apt支持https(已支持可忽略)

sudo apt-get install apt-transport-https ca-certificates curl software-properties-common

\item 添加Docker官方的GPG Key 

$curl -fsSL https://download.docker.com/linux/ubuntu/gpg | sudo apt-key add -$
\item 设置库
$sudo add-apt-repository "deb [arch=amd64] https://download.docker.com/linux/ubuntu (lsb_release -cs) stable"$

\item 更新 
sudo apt-get update

\item 安装Docker CE 
sudo apt-get install docker-ce

\item 安装Docker Compose 
sudo curl -L https://github.com/docker/compose/releases/download/1.21.0/docker-compose-(uname -s)-(uname -m) -o /usr/local/bin/docker-compose

\item 赋予权限 
sudo chmod +x /usr/local/bin/docker-compose

\item 免sudo执行docker(需重新登录shell) 

sudo usermod -aG docker {USER}
\end{enumerate}

\subsubsection*{配置docker加入overlay网络}
\begin{enumerate}
\item 修改docker启动参数 
sudo vim /lib/systemd/system/docker.service

\item 将ExecStart修改成如下:
ExecStart=/usr/bin/dockerd -H tcp://0.0.0.0:2376 -H unix:///var/run/docker.sock --cluster-store=consul://consul.azfs.com.cn:8500 --cluster-advertise=ens4:2376 --insecure-registry=0.0.0.0/0

\item 重启Docker Daemon 
sudo systemctl daemon-reload \\
sudo systemctl restart docker
\end{enumerate}
\subsubsection*{安装}
\begin{enumerate}
\item 下载配置文件 
git clone https://github.com/kyawn/deploy.git

\item 进入配置文件夹执行脚本 
cd deploy 
./parity-deploy.sh --config aura --name jnuchain

\item 运行容器 
docker-compose up -d

\item 停止并销毁容器 
docker-compose down

\item 开始|停止|重启某个服务 
docker-compose start | stop | restart + 服务名称(可选)

\item 查看日志 
docker-compose logs +服务名称(可选)
\end{enumerate}

\subsubsection*{加入节点}
\begin{enumerate}
\item 将新节点的$eNodeURL$加入到$reserved_peers$中

\item 将新节点的地址加入到spec.json中的validator和accounts中
\end{enumerate}

\subsubsection*{当前节点配置}
三个节点: 

node1.azfs.com.cn 35.194.230.35 
node2.azfs.com.cn 35.201.227.55
node3.azfs.com.cn 35.201.213.169

三个节点分别跑着parity和ipfs服务,并且node1运行monitor服务。

另外运行的consul服务地址为:http://consul.azfs.com.cn:8500
parity的RPC端口为8545,websocket端口为8546,UI端口为8180
例如:进入node1的parity UI为http://node1.azfs.com.cn:8180 (需要在后端生成token才能访问)

monitor服务地址为:http://node1.azfs.com.cn:3001

区块链浏览器 http://node2.azfs.com.cn:3000

\section{下周规划}
本周项目的事暂时告一段落,下周还是回到我之前看的PATE方面来,继续看PATE方面的文章。

\end{document}
