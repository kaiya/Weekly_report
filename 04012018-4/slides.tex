%!TEX program = xelatex
%!TEX TS-program = xelatex
%!TEX encoding = UTF-8 Unicode

\documentclass[aspectration=169]{beamer}
\usepackage[UTF8, heading = false, scheme = plain]{ctex}
\usepackage{graphicx}
\usepackage{cite}
% \usepackage{geometry}
% \geometry{left=2.0cm, right=2.0cm, top=2.5cm, bottom=2.5cm}
% \usepackage[colorlinks,linkcolor=red,anchorcolor=blue,citecolor=green]{hyperref}
\usepackage{subfig}
\usepackage{caption}
\captionsetup{font={scriptsize}}
\usepackage[T1]{fontenc}
\usepackage[square, authoryear, comma]{natbib}

\renewcommand\figurename{图}

%%%% 段落首行缩进两个字 %%%%
% \makeatletter
% \let\@afterindentfalse\@afterindenttrue
% \@afterindenttrue
% \makeatother
% \setlength{\parindent}{2em}  %中文缩进两个汉字位

%%%% 下面的命令设置行间距与段落间距 %%%%
\linespread{1.4}
% \setlength{\parskip}{1ex}
\setlength{\parskip}{0.5\baselineskip}

\title{学习汇报}
\author{熊凯亚}
\date{\today}
\institute[JNU]{Jinan University}
% \email{xiongkaiya@gmail.com}

\begin{document}

\begin{frame}
\titlepage
\end{frame}

\begin{frame}{Outline}
\tableofcontents[part=1, pausesections]
\end{frame}

\begin{frame}{引言}

\section{引言}

%最近两周看了一些关于去中心化深度学习中的隐私保护问题。如\cite{hitaj2017deep}中提出了一种攻击模型,这种模型针对于\cite{shokri2015privacy}中提出的协作学习进行了攻击。\cite{46432},\cite{abadi2016deep},\cite{bonawitz2017practical},\cite{Phong2017PrivacyPreservingDL}.

最近看了一些关于在去中心化的深度学习场景下的隐私保护的文章。去中心化的深度学习主要有两个方向,一是,CCS'15上Shokri等人\cite{shokri2015privacy}提出的隐私保护多方参与的协作深度学习模型(Collaborative Deep Learning);二是,Google 提出的针对移动设备(Android)的多方参与的\href{https://research.googleblog.com/2017/04/federated-learning-collaborative.html}{联合学习}(Federated Learning)。\\



\end{frame}


\begin{frame}{References}
\bibliographystyle{authordate1}
\bibliography{references}
\end{frame}
\end{document}


















